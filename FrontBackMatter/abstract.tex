%*******************************************************
% Abstract
%*******************************************************
%\renewcommand{\abstractname}{Abstract}
\pdfbookmark[1]{Résumé}{resume}
\begingroup
\let\clearpage\relax
\let\cleardoublepage\relax
\let\cleardoublepage\relax

\chapter*{Résumé}

La première partie de cette thèse est dédiée à l'extraction des corrections de l'échelle en énergie des jets dans l'expérience CMS du LHC. Les diverses méthodes d'extraction sont détaillées, en insistant tout particulièrement sur l'utilisation des événements photon + jets. Les corrections ainsi extraites sont primordiales pour les analyses de physique utilisant des jets, et sont d'ores et déjà utilisées par toute la collaboration.

La deuxième partie présente la recherche de physique au-delà du Modèle Standard dans le secteur du quark top. De nombreux modèles de nouvelle physique prédisent en effet de nouvelles particules, pouvant se manifester comme des résonances dans le spectre de masse top-antitop. Deux analyses sont présentées dans cette thèse, une dédiée à la recherche des résonances de spin 1, publiée dans Physical Review Letters, et une dédiée à la recherche de résonances de spin 0, où les phénomènes d'interférences avec le Modèle Standard sont pris en compte.

\vfill

\pdfbookmark[1]{Abstract}{abstract}
\chapter*{Abstract}

\fxnote{Traduire}
La première partie de cette thèse est dédiée à l'extraction des corrections de l'échelle en énergie des jets dans l'expérience CMS du LHC. Les diverses méthodes d'extraction sont détaillées, en insistant tout particulièrement sur l'utilisation des événements photon + jets. Les corrections ainsi extraites sont primordiales pour les analyses de physique utilisant des jets, et sont d'ores et déjà utilisées par toute la collaboration.

La deuxième partie présente la recherche de physique au-delà du Modèle Standard dans le secteur du quark top. De nombreux modèles de nouvelle physique prédisent en effet de nouvelles particules, pouvant se manifester comme des résonances dans le spectre de masse top-antitop. Deux analyses sont présentées dans cette thèse, une dédiée à la recherche des résonances de spin 1, publiée dans Physical Review Letters, et une dédiée à la recherche de résonances de spin 0, où les phénomènes d'interférences avec le Modèle Standard sont pris en compte.

\endgroup

\vfill
