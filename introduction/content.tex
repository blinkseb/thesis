\chapter*{Introduction}
\addcontentsline{toc}{chapter}{Introduction}


La physique des particules décrit les constituants fondamentaux de la matière, ainsi que leurs interactions, au sein d'une théorie quantique des champs, le Modèle Standard. Les nombreux succès du modèle depuis son élaboration dans les années 60 en font une description rigoureuse de la Nature, jamais mis en défaut aux travers des nombreux tests effectués. Le succès le plus récent, et probablement le plus marquant, est la découverte du boson de Higgs, particule prédite depuis 1964 et observée pour la première fois en 2012.

Bien que jamais mis en défaut, le Modèle Standard ne parvient pas à expliquer tous les comportements de la Nature. De nombreux arguments laissent penser que ce modèle n'est qu'une théorie effective d'un modèle plus fondamentale se manifestant à plus haute énergie. Tout d'abord, la gravitation n'est pas incluse dans la description des interactions fondamentales. La découverte dans les années 2000 de l'oscillation des neutrinos impose qu'ils soient massifs. Même si le Modèle Standard peut s'en accommoder, ces masses ne sont pas prédites dans la version simple du modèle. D'autre part, la matière noire et l'énergie noire, composant près de \SI{95}{\percent} de la densité d'énergie de l'Univers, restent totalement inexpliquées. Face à ces lacunes, de nombreuses théories ont été développées pour étendre le Modèle Standard et proposer une explication théorique à ces phénomènes. Parmi ces théories, de nombreuses prévoient de nouvelles particules ayant un fort couplage avec le quark top. Découvert en 1995, c'est en effet la particule fondamentale la plus lourde, laissant supposer que le quark top joue un rôle majeur dans la recherche de nouvelle physique. À ce jour, aucune de ces théories n'a encore été confirmée.

\smallskip

C'est dans ce contexte que le LHC a été conçu. Ce puissant accélérateur permet d'atteindre des énergies jamais atteintes dans des collisions de particules, et ainsi sonder la matière à des échelles inédites et tester en profondeur le Modèle Standard. Le programme scientifique du LHC est assuré par quatre expériences, dont deux expériences généralistes, ATLAS et CMS. Depuis la découverte du boson de Higgs, les recherches se focalisent principalement sur la mise en évidence de nouvelles particules ou de toutes autres signatures de nouvelle physique, mais également sur les mesures de précisions des paramètres du Modèle Standard.

\bigskip

C'est au sein de la collaboration CMS que le travail présenté dans cette thèse a été effectué. Le premier chapitre présente le Modèle Standard, ses forces et ses faiblesses. Les deuxième et troisième chapitres sont consacrés à la description du dispositif expérimental, le LHC et le détecteur CMS, ainsi qu'aux techniques de simulation et de reconstruction des événements. Le chapitre 4 présente les méthodes de calibration de l'échelle en énergie des jets, plus spécifiquement la dérivation des corrections en utilisant des événements $\gamma$ + jets.

\medskip

Les chapitres suivants sont dédiés à la recherche de nouvelle physique dans le secteur du quark top. Le chapitre 5 présente en détail le quark top, ses modes de production et de désintégration, ses propriétés, mais surtout son lien avec divers modèles de nouvelle physique. Le sixième chapitre est dédié à la reconstruction des événements \ttbar, permettant la recherche de physique au-delà du Modèle Standard. Le chapitre 7 se concentre sur la recherche de nouvelles particules de spin 1, les \zprime et les gluons de Kaluza-Klein. Le huitième et dernier chapitre est quant à lui dédié à la recherche de nouvelles particules de spin 0, assimilables à des bosons de Higgs massifs. Cette analyse novatrice prend en compte les effets d'interférence avec la production du Modèle Standard, entraînant plusieurs difficultés, de la génération du signal jusqu'à la stratégie d'analyse.