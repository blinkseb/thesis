% XeTeX configuration
% All code inspired by Classic Thesis package, and customization by arsclassica

% Flags
\RequirePackage{ifthen}
    \newboolean{@drafting} % print version information on pages
    \newboolean{@tocaligned} % the left column of the toc will be aligned (no indention)
    \newboolean{@eulerchapternumbers} % use AMS Euler for chapter font (otherwise Palatino)
    \newboolean{@linedheaders} % chaper headers will have line above and beneath
    \newboolean{@listsseparated} % toggles the vertical space between lof/lot entries of different chapters
    \newboolean{@nochapters} % disable all chapter-specific commands 
    \newboolean{@beramono} % toggle nice monospaced font (w/ bold) + pre-installed 
    \newboolean{@eulermath} % use awesome Euler fonts for math
    \newboolean{@parts} % use part division for the text
    \newboolean{@minionpro} % setup for minion pro font
    \newboolean{@minionprospacing} % use minion pro's textssc for letter spacing
    \newboolean{@pdfspacing} % use pdftex for letterspacing (via microtype)
    \newboolean{@a5paper} % use those tiny DIN A5 pages
    \newboolean{@dottedtoc} % page numbers in ToC flushed right
  	\newboolean{@listings} % load listings package (if not already) and setup LoL
	  \newboolean{@manychapters} % additional space in ToC after chapter number (if two digits are needed)
	  \newboolean{@floatperchapter} % numbering per chapter for all floats (i.e., Figure 1.1)

\setboolean{@drafting}{false}

\newboolean{linenumbers} % enable backrefs in the bibliography
\setboolean{linenumbers}{true}

\newboolean{enable-backrefs} % enable backrefs in the bibliography
\setboolean{enable-backrefs}{true} % true false

\usepackage{polyglossia}
\setdefaultlanguage{french}
\usepackage[autostyle,french=guillemets*]{csquotes}
\MakeOuterQuote{"}
\frenchspacing

\usepackage{fontspec}

% Main font to Palatino
\setmainfont
     [Ligatures={Common}, Fractions=On, Mapping=tex-text,
      BoldFont       = texgyrepagella-bold.otf,
      ItalicFont     = texgyrepagella-italic.otf,
      BoldItalicFont = texgyrepagella-bolditalic.otf]{texgyrepagella-regular.otf}

% Lato font
% \setmainfont
%      [Ligatures={Common}, Fractions=On, Mapping=tex-text,
%       Path           = fonts/,
%       Extension      = .otf,
%       UprightFont    = *-Reg,
%       BoldFont       = *-Bol,
%       ItalicFont     = *-RegIta,
%       BoldItalicFont = *-BolIta]{Lato}
      
\setsansfont[
    Ligatures={Common}, Mapping=tex-text,
    Extension=.otf,
    UprightFont= *-Regular,
    BoldFont=*-Bold,
    ItalicFont=*-Italic,
    BoldItalicFont=*-BoldItalic,
    Numbers={Uppercase}
]{Iwona}

\usepackage[protrusion=true]{microtype} % character protruding and other micro-typography stuff
\RequirePackage{amsmath} % ``Proper'' math commands and environments
\RequirePackage{braket} % Dirac bra-ket notation
\RequirePackage{slashed} % also for slash notation: take your pick!
\RequirePackage{hepnicenames} % A ``friendly'' macro set for selected PEN symbols 
                              % which includes hepparticles and heppennames

\RequirePackage{siunitx}
\sisetup{locale = FR}

\usepackage{unicode-math}
\setmathfont[Path=fonts/]{texgyrepagella-math.otf}
\setmathfont[range={\mathcal,\mathbfcal},StylisticSet=1]{XITS Math}
%\setmathfont{Asana Math}

\setmonofont[
    Ligatures={Common}, Mapping=tex-text,
    BoldFont=SourceCodePro-Bold.otf,
    Path=fonts/
]{SourceCodePro-Regular.otf}

\linespread{1.05}

% Font for chapter numbers
\newfontface{\chapterNumber}[Scale=4.5,Path=fonts/]{euler.otf}
\newfontface{\chapterName}[Scale=4,Path=fonts/]{euler.otf}

\newcounter{dummy} % Necessary for correct hyperlinks (to index, bib, etc.)

% Bibliography
\PassOptionsToPackage{square,numbers}{natbib}
\usepackage{natbib}

\usepackage{lipsum}

%\usepackage{listings}
\usepackage{tabularx} % Better tables

\PassOptionsToPackage{dvipsnames}{xcolor}
\RequirePackage{xcolor}
\definecolor{halfgray}{gray}{0.55} % chapter numbers will be semi transparent .5 .55 .6 .0
\definecolor{webgreen}{rgb}{0,.5,0}
\definecolor{webbrown}{rgb}{.6,0,0}

%\definecolor{darkblue}{HTML}{054D73} % URL colors
%\definecolor{darkblue}{HTML}{044262} % URL colors
\definecolor{darkblue}{HTML}{033651} % URL colors

\definecolor{chapter_in_toc}{HTML}{8B1B26}

% \makeatletter
% \let\latex@include\include
% \def\include#1{\include@aux#1\@nil}
% \def\include@aux#1/#2\@nil{\@mkdir{#1}\latex@include{#1/#2}}
% %\def\include@aux#1/#2\@nil{\latex@include{#1/#2}}
% \ifnum\shellescape=\@ne
%   \def\@mkdir#1{\immediate\write18{mkdir -p output/#1}}
% \else
%   \def\@mkdir{\typeout{Expect errors}}
% \fi
% \makeatother


\usepackage{graphicx}
\graphicspath{{figs/}{logos/}{chapitre1/figs/}{chapitre2/figs/}}

% Disable single lines at the start of a paragraph (Schusterjungen)
\clubpenalty = 10000
% Disable single lines at the end of a paragraph (Hurenkinder)
\widowpenalty = 10000 
\displaywidowpenalty = 10000 % formulas

% Graffiti as in GKP's book "Concrete Mathematics"
% thanks to Lorenzo Pantieri and Enrico Gregorio
\def\graffito@setup{%
   \slshape\footnotesize%
   \parindent=0pt \lineskip=0pt \lineskiplimit=0pt %
   \tolerance=2000 \hyphenpenalty=300 \exhyphenpenalty=300%
   \doublehyphendemerits=100000%
   \finalhyphendemerits=\doublehyphendemerits}
%\DeclareRobustCommand{\graffito}[1]{\marginpar%
% [\graffito@setup\raggedleft\hspace{0pt}{#1}]%
% {\graffito@setup\raggedright\hspace{0pt}{#1}}}
\let\oldmarginpar\marginpar
\renewcommand{\marginpar}[1]{\oldmarginpar%
 [\graffito@setup\raggedleft\hspace{0pt}{#1}]%
 {\graffito@setup\raggedright\hspace{0pt}{#1}}}
 
 \RequirePackage{xfrac}
\RequirePackage{booktabs} % for better rules in tables
\RequirePackage{multirow}
\RequirePackage{textcase} % for \MakeTextUppercase

\newcommand\xetextls[2]{{\addfontfeature{LetterSpace=#1}#2}}
%\DeclareRobustCommand{\spacedallcaps}[1]{\MakeTextUppercase{#1}}%
\DeclareRobustCommand{\spacedallcaps}[1]{\sffamily\xetextls{10}{\MakeTextUppercase{#1}}}%
\DeclareRobustCommand{\spacedlowsmallcaps}[1]{\sffamily\xetextls{10}{\scshape\MakeTextLowercase{#1}}}%
\DeclareRobustCommand{\lessspacedlowsmallcaps}[1]{\sffamily\xetextls{5}{\scshape\MakeTextLowercase{#1}}}%

% ********************************************************************                
% headlines
% ************************************************************
\PassOptionsToPackage{automark}{scrpage2}
	\RequirePackage{scrpage2} % provides headers and footers (KOMA Script)
    \clearscrheadings
    \setheadsepline{0pt}
    
    \renewcommand{\chaptermark}[1]{\markboth{\spacedlowsmallcaps{#1}}{\spacedlowsmallcaps{#1}}}
    \renewcommand{\sectionmark}[1]{\markright{\textsc{\MakeTextLowercase{\thesection}}\enspace\spacedlowsmallcaps{#1}}}
    
    \lehead{\mbox{\llap{\small\thepage\kern1em\color{halfgray}\vline}\color{halfgray}\hspace*{0.5em}\headmark\hfil}}
    \rohead{\mbox{\hfil{\color{halfgray}\headmark\hspace{0.5em}}\rlap{\small{\color{halfgray}\vline}\kern1em\thepage}}}
    
    \renewcommand{\headfont}{\small}  
    
    \renewcommand{\headfont}{\normalfont\sffamily}
    \renewcommand{\pnumfont}{\small\sffamily}
%    \DeclareRobustCommand{\fixBothHeadlines}[2]{} % <--- ToDo
    % hack to get the content headlines right (thanks, Lorenzo!)
		\def\toc@heading{%
	 		\chapter*{\contentsname}%chapters
	 		\@mkboth{\spacedlowsmallcaps{\contentsname}}{\spacedlowsmallcaps{\contentsname}}}
	 		
	 		
	 		
% ********************************************************************
% layout of the chapter-, section-, subsection-, subsubsection-,
% paragraph and description-headings
% ********************************************************************  
\RequirePackage{titletoc}
\RequirePackage[calcwidth]{titlesec}
    \makeatletter
    \patchcmd{\ttl@select}{\strut}{}{}{}
    \patchcmd{\ttlh@display}{\strut}{}{}{}
    \patchcmd{\ttlh@display}{\strut}{}{}{}
    \patchcmd{\ttlh@hang}{\strut}{}{}{}
    \patchcmd{\ttlh@hang}{\strut}{}{}{}
    \patchcmd{\ttlh@runin}{\strut}{}{}{}
    \makeatother

%    \titleformat{\chapter}[display]%
%          {\relax}{\mbox{}\oldmarginpar{\vspace*{-1.35\baselineskip}\color{halfgray}\chapterNumber\thechapter}}{0pt}%
%          {\begingroup\setlength{\baselineskip}{1.35\baselineskip}\raggedright\headingFont\addfontfeature{Scale=1.4}\spacedallcaps}[\endgroup\normalsize\vspace*{.8\baselineskip}\titlerule\vspace*{1pc}\startcontents\printcontents{mini}{1}{}]%
%    \titleformat{\chapter}[display]%
%          {\relax}{\raggedleft{\color{halfgray}\chapterName\chaptername\chapterNumber\thechapter}}{10pt}%
%          {\raggedright\headingFont\addfontfeature{Scale=1.4}\spacedallcaps}[\normalsize\vspace*{.8\baselineskip}\titlerule%
%          \vspace*{1pc}\startcontents\printcontents{}{1}{}]%

%    \titleformat{\chapter}[hang]%
%          {\relax}{\thechapter}{0pt}%
%          {\headingFont\addfontfeature{Scale=1.4}}[\normalsize\vspace*{.8\baselineskip}\titlerule\vspace*{1pc}]%

%     % sections \FloatBarrier
%     \titleformat{\section}
%         {\relax}{\headingFont\addfontfeature{Scale=1.25}\thesection}{1em}{\headingFont\addfontfeature{Scale=1.25}}
%     % subsections
%     \titleformat{\subsection}
%         {\relax}{\headingFont\addfontfeature{Scale=1.10}\thesubsection}{1em}{\headingFont\addfontfeature{Scale=1.10}}
%     % subsubsections
%     \titleformat{\subsubsection}
%         {\relax}{\headingFont\thesubsubsection}{1em}{\headingFont\bfseries}        
%     % paragraphs
%     \titleformat{\paragraph}[runin]
%         {\normalfont\normalsize}{\theparagraph}{0pt}{\spacedlowsmallcaps}    
%     % descriptionlabels
%         \renewcommand{\descriptionlabel}[1]{\hspace*{\labelsep}\spacedlowsmallcaps{#1}}   % spacedlowsmallcaps textit textsc                  


% \renewcommand{\tabularxcolumn}[1]{b{#1}}%
% \newcommand\formatchapter[1]{% 
% \begin{tabularx}{\titlewidth}[b]{@{} X @{}} 
%   \filright\spacedallcaps{#1}
% \end{tabularx}}

\titleformat{\chapter}[block]{\normalfont\Large\sffamily}{{\color{halfgray}\chapterNumber\thechapter\hspace{10pt}\vline}}{10pt}{\formatchapter}
\titleformat{\section}{\normalfont\Large\sffamily}{\textsc{\MakeTextLowercase{\thesection}}}{1em}{\lineskiplimit=-10pt\linespread{0.8}\spacedlowsmallcaps}
%\titleformat{\section}{\normalfont\Large\sffamily}{\textsc{\MakeTextLowercase{\thesection}}}{1em}{\linespread{0.8}\spacedlowsmallcaps}

% subsections
\titleformat{\subsection}{\normalfont\sffamily}{\textsc{\MakeTextLowercase{\thesubsection}}}{1em}{\normalsize}
% subsubsections
\titleformat{\subsubsection}{\normalfont\sffamily\itshape}{\textsc{\MakeTextLowercase{\thesubsubsection}}}{1em}{\normalsize\itshape}

\titleformat{\paragraph}[hang]{}{\theparagraph}{0pt}{\bfseries}

% descriptionlabels
\renewcommand{\descriptionlabel}[1]{\hspace*{\labelsep}\bfseries\lessspacedlowsmallcaps{#1}}

\titlespacing*{\chapter}{0pt}{1\baselineskip}{2\baselineskip}
\titlespacing*{\section}{0pt}{2\baselineskip}{.8\baselineskip}[\marginparsep]
\titlespacing*{\subsection}{0pt}{1.5\baselineskip}{.8\baselineskip}[\marginparsep]
\titlespacing*{\subsubsection}{0pt}{3.25ex plus 1ex minus .2ex}{1.5ex plus .2ex}
\titlespacing*{\paragraph}{0pt}{1\baselineskip}{1\baselineskip}

%     \newcommand\formatchapter[1]{%
%     \vbox to \ht\strutbox{
%     \setbox0=\hbox{\chapterNumber\thechapter\hspace{10pt}\vline\ } 
%     \advance\hsize-\w\dzero \advance\hsize-10pt\raggedright 
%     \spacedallcaps{#1}\vss}} 
%     \titleformat{\chapter}[block] 
%        {\normalfont\Large\sffamily} 
%        {\textcolor{halfgray}{\chapterNumber\thechapter} 
%        \hspace{10pt}\vline\ }{10pt} 
%     {\formatchapter}
  

  \renewcommand{\tabularxcolumn}[1]{b{#1}}

  \newcommand\formatchapter[1]{%
  \begin{tabularx}{\linewidth}{@{}l@{}@{\hspace{20pt}}X@{}} 
       \textcolor{halfgray}{\chapterNumber\thechapter\hspace*{8pt}\vline} & \spacedallcaps{#1}
  \end{tabularx}}
  
  \newcommand\formatchapternonumber[1]{%
  \begin{tabularx}{\linewidth}{@{}X@{}} 
       \spacedallcaps{#1}
  \end{tabularx}}
  
  \titleformat{name=\chapter}[block]{\normalfont\Large\sffamily}{\relax}{0pt}{\formatchapter}[\normalsize\vspace*{.8\baselineskip}\vspace*{1pc}\startcontents\printcontents{mini}{1}{}]

  \titleformat{name=\chapter,numberless}[block]{\normalfont\Large\sffamily}{\relax}{0pt}{\formatchapternonumber}[\normalsize\vspace*{.8\baselineskip}\titlerule]
  
	\makeatletter
	\if@twoside\rofoot[\mbox{\makebox[0pt][l]{\kern1em\thepage}}]{}\fi
	\makeatother

%************************************************************
% itemize
%************************************************************
\renewcommand\labelitemi{\color{halfgray}$\bullet$} 

%************************************************************
% caption
%************************************************************
\usepackage{newfloat}
\DeclareFloatingEnvironment[fileext=eqs,placement={tbp},name=Éq.]{eq}

\usepackage{subcaption}
\DeclareCaptionSubType{figure}
\DeclareCaptionSubType{table}
\DeclareCaptionSubType{eq}

\captionsetup{format=plain,indention=0.2cm,font=small,labelfont={sf,bf}}
\captionsetup[eq]{format=plain,indention=0.2cm,font=small,labelfont={sf,bf}}

\makeatletter
\let\c@eq\c@equation
\makeatother

% ********************************************************************
% footnotes setup   
% ********************************************************************   
\ifdefined\deffootnote
  % KOMA-command, footnotemark not superscripted at the bottom
  \deffootnote{0em}{0em}{\thefootnotemark\hspace*{.5em}}%
  \message{Using KOMA-command "deffootnote" for footnote setup}%
\else
	\PassOptionsToPackage{flushmargin}{footmisc}%
	\RequirePackage{footmisc}%
	\setlength{\footnotemargin}{-.5em}%
  	\PackageWarningNoLine{classicthesis}{Using package "footmisc" with option %
  		"flushmargin" for footnote setup (not 100\% the same as with KOMA)}% 
\fi

\PassOptionsToPackage{titles}{tocloft}
	\RequirePackage{tocloft}

% ********************************************************************
% Drafting Stuff
% ********************************************************************
\usepackage{xspace} % To get the spacing after macros right
\newcommand{\myVersion}{version 4.0\xspace}

\RequirePackage{scrtime} % time access
\newcommand{\finalVersionString}{}
 \ifthenelse{\boolean{@drafting}}{% 
% %    \RequirePackage{draftwatermark}%
% %    	\SetWatermarkLightness{0.9}
% %			\SetWatermarkScale{.5}
% %			\SetWatermarkText{\today\ at \thistime}
 		\PassOptionsToPackage{draft}{prelim2e}
 		\RequirePackage{prelim2e}
       \renewcommand{\PrelimWords}{Version préliminaire}
     %    \renewcommand{\PrelimText}{\myVersion}
 }{\relax}

\usepackage{fixltx2e} % fixes some LaTeX stuff 
\PassOptionsToPackage{smaller}{acronym}
	\usepackage{acronym} % nice macros for handling all acronyms in the thesis
\renewcommand{\bflabel}[1]{{#1}\hfill} % fix the list of acronyms

\setlength{\extrarowheight}{3pt} % Increase table row height
\newcommand{\tableheadline}[1]{\multicolumn{1}{c}{\spacedlowsmallcaps{#1}}}
\newcommand{\myfloatalign}{\centering} % To be used with each float for alignment

\PassOptionsToPackage{hyperfootnotes=false,pdfpagelabels}{hyperref}
\usepackage{hyperref}  % backref linktocpage pagebackref
%\pdfcompresslevel=9
%\pdfadjustspacing=1

% Back references

\newcommand{\backrefnotcitedstring}{\relax}%(Not cited.)
\newcommand{\backrefcitedsinglestring}[1]{(Cité en page~#1)}
\newcommand{\backrefcitedmultistring}[1]{(Cité en pages~#1)}
\ifthenelse{\boolean{enable-backrefs}}%
{%
		\PassOptionsToPackage{hyperpageref}{backref}
		\usepackage{backref} % to be loaded after hyperref package 
		   \renewcommand{\backreftwosep}{ et~} % separate 2 pages
		   \renewcommand{\backreflastsep}{, et~} % separate last of longer list
		   \renewcommand*{\backref}[1]{}  % disable standard
		   \renewcommand*{\backrefalt}[4]{% detailed backref
		      \ifcase #1 %
		         \backrefnotcitedstring%
		      \or%
		         \backrefcitedsinglestring{#2}%
		      \else%
		         \backrefcitedmultistring{#2}%
		      \fi}%
}{\relax}

\hypersetup{
 % Uncomment the line below to remove all links (to references, figures, tables, etc)
 %draft, 
 colorlinks=true, linktocpage=false, pdfstartpage=3, pdfstartview=FitV,
 % Uncomment the line below if you want to have black links (e.g. for printing black and white)
 %colorlinks=false, linktocpage=false, pdfborder={0 0 0}, pdfstartpage=3, pdfstartview=FitV, 
 breaklinks=true, pdfpagemode=UseNone, pageanchor=true, pdfpagemode=UseOutlines,
 plainpages=false, bookmarksnumbered, bookmarksopen=true, bookmarksopenlevel=1,
 hypertexnames=true, pdfhighlight=/O, urlcolor=webbrown, linkcolor=darkblue, citecolor=webgreen,
 %------------------------------------------------
 % PDF file meta-information
 pdftitle={Recherche de physique au-delà du Modèle Standard dans le secteur du quark top et calibration de l'échelle en énergie des jets avec l'expérience CMS du LHC},
 pdfauthor={Sébastien Brochet},
 pdfsubject={},
 pdfkeywords={LHC,CMS,quark top,topcolor,higgs,Z',new physics},
 pdfcreator={XeTeX}
}

\contentsmargin{2.5em}

% Mini-toc style
\titlecontents{minisection}[2.3em]{\color{darkblue}}{\contentslabel{2.3em}}{\hspace*{-2.3em}}{\titlerule*[0.8pc]{.}\contentspage}
\titlecontents{minisubsection}[5.5em]{\color{darkblue}}{\contentslabel{3.2em}}{\hspace*{-3.2em}}{\titlerule*[0.8pc]{.}\contentspage}

% Main toc style

\titlecontents{chapter}[1.3em]{\addvspace{1pc}\hypersetup{linkcolor=chapter_in_toc}\color{chapter_in_toc}}{\contentslabel{1.3em}}{\hspace*{-1.3em}}{\titlerule*[0.8pc]{.}\contentspage}
\titlecontents{section}[3.2em]{\addvspace{0.6pc}\hypersetup{linkcolor=black}}{\contentslabel{1.9em}}{\hspace*{-1.9em}}{\titlerule*[0.8pc]{.}\contentspage}
\titlecontents{subsection}[5.8em]{\addvspace{0.2em}\hypersetup{linkcolor=black}}{\contentslabel{2.6em}}{\hspace*{0em}}{\titlerule*[0.8pc]{.}\contentspage}

% Reset footnote counter each time a new chapter starts
\makeatletter
\pretocmd{\@schapter}{\setcounter{footnote}{0}}{}{}
\pretocmd{\@chapter}{\setcounter{footnote}{0}}{}{}
\makeatother

\usepackage{feynmp}
\DeclareGraphicsRule{*}{eps}{*}{}

% Usefull macro
% \newcommand{\MeVc}{\MeVoverc}
% \newcommand{\MeVcc}{\MeVovercsq}
% \newcommand{\GeVc}{\GeVoverc}
% \newcommand{\GeVcc}{\GeVovercsq}
% \newcommand{\TeVc}{\TeVoverc}
% \newcommand{\TeVcc}{\TeVovercsq}

\newcommand{\lumi}{\ensuremath{\mathcal{L}}\xspace}
\newcommand{\Lumi}{\ensuremath{\mathcal{L}}\xspace} % both upper and lower

\newcommand{\PT}{\ensuremath{p_{\mathrm{T}}}\xspace}
\newcommand{\pt}{\PT}
\newcommand{\ET}{\ensuremath{E_{\mathrm{T}}}\xspace}
\newcommand{\et}{\ET}
\newcommand{\HT}{\ensuremath{H_{\mathrm{T}}}\xspace}
\newcommand{\Em}{\ensuremath{\slashed{E}}\xspace}
\newcommand{\Pm}{\ensuremath{\slashed{p}}\xspace}
\newcommand{\PTm}{\ensuremath{\slashed{p}_\mathrm{T}}\xspace}
\newcommand{\PTslash}{\PTm}
\newcommand{\ETm}{\ensuremath{E_{\mathrm{T}}^{\text{miss}}}\xspace}
\newcommand{\ETmiss}{\ETm}
\newcommand{\ETslash}{\ensuremath{\slashed{E}_{\mathrm{T}}}\xspace}
\newcommand{\MET}{\ETslash}
\newcommand{\met}{\MET}
\newcommand{\METx}{\ensuremath{\slashed{E}_{\mathrm{x}}}\xspace}
\newcommand{\METy}{\ensuremath{\slashed{E}_{\mathrm{y}}}\xspace}
\newcommand{\VEtmiss}{\ensuremath{{\vec E}_{\mathrm{T}}^{\text{miss}}}\xspace}
\newcommand{\ptvec}{\ensuremath{{\vec p}_{\mathrm{T}}}\xspace}

\newcommand{\abs}[1]{\ensuremath{\lvert #1 \rvert}}
\newcommand{\norm}[1]{\ensuremath{\left\lVert \, #1 \, \right\rVert}}

\newcommand{\ttbar}{\texorpdfstring{\ensuremath{\Ptop\APtop}\xspace}{top-antitop\xspace}}
\newcommand{\bbbar}{\ensuremath{\Pbottom\APbottom}\xspace}
\newcommand{\mtt}{\ensuremath{m_{\Pqt\Paqt}}\xspace}
\newcommand{\mt}{\ensuremath{m_{\Pqt}}\xspace}
\newcommand{\zprime}{\ensuremath{\PZ^{\prime}}\xspace}
\newcommand{\mzp}{\ensuremath{m_{\PZ^{\prime}}}\xspace}
\newcommand{\msz}{\ensuremath{m_{\sz}}\xspace}
\newcommand{\kkglu}{gluon K. K.\xspace}
\newcommand{\kkg}{\ensuremath{\Pgluon^\text{(1)}}\xspace}
\newcommand{\sz}{\HepParticle{S}{0}{}\xspace}

\newcommand{\gmu}{\ensuremath{\gamma_\mu}}
\newcommand{\gzero}{\ensuremath{\gamma_0}}
\newcommand{\gfive}{\ensuremath{\gamma_5}}
\newcommand{\dmu}{\ensuremath{\partial_\mu}}

\newcommand{\gmuc}{\ensuremath{\gamma^\mu}}
\newcommand{\gzeroc}{\ensuremath{\gamma^0}}
\newcommand{\gfivec}{\ensuremath{\gamma^5}}
\newcommand{\dmuc}{\ensuremath{\partial^\mu}}

\newcommand{\Psib}{\ensuremath{\bar{\Psi}}}

\newcommand{\pu}{\emph{pile-up}\xspace}
\newcommand{\pf}{\emph{particle-flow}\xspace}

\newcommand{\aeta}{\ensuremath{\abs{\eta}}\xspace}

% High energy physics
\DeclareSIUnit\c{c}
\DeclareSIUnit\micron{\micro\metre}
\DeclareSIUnit\mrad{\milli\rad}
\DeclareSIUnit\gauss{G}
\DeclareSIUnit\eVperc{\eV\per\clight}
\DeclareSIUnit\nanobarn{\nano\barn}
\DeclareSIUnit\picobarn{\pico\barn}
\DeclareSIUnit\femtobarn{\femto\barn}
\DeclareSIUnit\attobarn{\atto\barn}
\DeclareSIUnit\zeptobarn{\zepto\barn}
\DeclareSIUnit\yoctobarn{\yocto\barn}
\DeclareSIUnit\nb{\nano\barn}
\DeclareSIUnit\pb{\pico\barn}
\DeclareSIUnit\fb{\femto\barn}
\DeclareSIUnit\ab{\atto\barn}
\DeclareSIUnit\zb{\zepto\barn}
\DeclareSIUnit\yb{\yocto\barn}

\DeclareSIUnit\invpb{\per\pico\barn}
\DeclareSIUnit\invfb{\per\femto\barn}

\DeclareSIUnit\GeVc{\GeV\per\c}
\DeclareSIUnit\GeVcc{\GeV\per\square\c}
\DeclareSIUnit\TeVc{\TeV\per\c}
\DeclareSIUnit\TeVcc{\TeV\per\square\c}

\DeclareSIUnit\octet{io}

% Column matrix
\newcount\colveccount
\newcommand*\colvec[1]{
        \global\colveccount#1
        \begin{pmatrix}
        \colvecnext
}
\def\colvecnext#1{
        #1
        \global\advance\colveccount-1
        \ifnum\colveccount>0
                \\
                \expandafter\colvecnext
        \else
                \end{pmatrix}
        \fi
}

% Row matrix
\def\and{&\xspace}
\newcount\rowveccount
\newcommand*\rowvec[1]{
        \global\rowveccount#1
        \begin{pmatrix}
        \rowvecnext
}
\def\rowvecnext#1{
        #1
        \global\advance\rowveccount-1
        \ifnum\rowveccount>0
                \and
                \expandafter\rowvecnext
        \else
                \end{pmatrix}
        \fi
}

% Tilde
\renewcommand{\tilde}{{\raise.17ex\hbox{$\scriptstyle\sim$}}}


\ifthenelse{\boolean{@drafting}}{%
\PassOptionsToPackage{status=draft}{fixme}
 }{\relax}
\RequirePackage[layout=inline]{fixme}

\usepackage{cleveref}

\usepackage{tikz}

\usepackage{multicol}

\usepackage{longtable}

\usepackage{fmtcount}
\FCloadlang{frenchb}
\fmtcountsetoptions{french=france}
\fmtcountsetoptions{abbrv=true}

\newcommand{\ptg}{\ensuremath{p_T^{\Pphoton}}\xspace}
\newcommand{\ptfjet}{\ensuremath{p_T^{\text{\ordinalnum{1} jet}}}\xspace}
\newcommand{\ptsjet}{\ensuremath{p_T^{\text{\ordinalnum{2} jet}}}\xspace}
\newcommand{\ptjet}{\ensuremath{p_T^{\text{jet}}}\xspace}
\newcommand{\chid}{\ensuremath{\chi^{2}}\xspace}
\newcommand{\mttgen}{\ensuremath{\mtt^\text{gen}}}
\newcommand{\pvalue}{\emph{p-value}\xspace}
\newcommand{\pvalues}{\emph{p-values}\xspace}

\usepackage{afterpage}

\definecolor{bleu_gris}{HTML}{556270}
\definecolor{bleu}{HTML}{0B486B}
\definecolor{rouge_grandmere}{HTML}{C44D58}
\definecolor{orange}{HTML}{FA6900}
\definecolor{orange2}{HTML}{E97F02}
\definecolor{violet}{HTML}{542437}
\definecolor{vert}{HTML}{8A9B0F}

\newcommand{\rouge}{\textcolor{rouge_grandmere}{rouge}\xspace}
\newcommand{\vertc}{\textcolor{vert}{vert}\xspace}
\newcommand{\verte}{\textcolor{vert}{verte}\xspace}
\newcommand{\gris}{\textcolor{bleu_gris}{gris}\xspace}
\newcommand{\grise}{\textcolor{bleu_gris}{grise}\xspace}
\newcommand{\bleu}{\textcolor{bleu}{bleu}\xspace}
\newcommand{\violet}{\textcolor{violet}{violet}\xspace}
\newcommand{\violette}{\textcolor{violet}{violette}\xspace}
\newcommand{\orange}{\textcolor{orange}{orange}\xspace}

\newcommand{\oranged}{\textcolor{orange2}{orange}\xspace}

\newcommand{\dzero}{DØ\xspace}

\usepackage{footnote}

%\setlength{\rightskip}{10cm}
%\AtNextShipout{\global\rightskip=0pt}

\usepackage[shortcuts]{extdash}

\crefname{eq}{équation}{équations}
\crefformat{equation}{équation~#2#1#3}

\newcolumntype{L}[1]{>{\raggedright\let\newline\\\arraybackslash\hspace{0pt}}m{#1}}
\usepackage[absolute]{textpos}
\setlength{\TPHorizModule}{\textwidth} % échelle horizontale
\setlength{\TPVertModule}{\textheight} % échelle verticale identique à l'horizontale

\ifthenelse{\boolean{linenumbers}}{% 
\usepackage[switch]{lineno}
\setlength\linenumbersep{1cm}
% Remove this when line numbers are not shown
\newcommand*\patchAmsMathEnvironmentForLineno[1]{%
  \expandafter\let\csname old#1\expandafter\endcsname\csname #1\endcsname
  \expandafter\let\csname oldend#1\expandafter\endcsname\csname end#1\endcsname
  \renewenvironment{#1}%
     {\linenomath\csname old#1\endcsname}%
     {\csname oldend#1\endcsname\endlinenomath}}% 
\newcommand*\patchBothAmsMathEnvironmentsForLineno[1]{%
  \patchAmsMathEnvironmentForLineno{#1}%
  \patchAmsMathEnvironmentForLineno{#1*}}%
\AtBeginDocument{%
\patchBothAmsMathEnvironmentsForLineno{equation}%
\patchBothAmsMathEnvironmentsForLineno{align}%
\patchBothAmsMathEnvironmentsForLineno{flalign}%
\patchBothAmsMathEnvironmentsForLineno{alignat}%
\patchBothAmsMathEnvironmentsForLineno{gather}%
\patchBothAmsMathEnvironmentsForLineno{multline}%
}
}{\relax}