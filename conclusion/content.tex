\chapter*{Conclusion}
\addcontentsline{toc}{chapter}{Conclusion}

Les travaux effectués dans cette thèse au sein de la collaboration CMS s'articulent autour de deux axes. Le premier axe est dédié aux corrections en énergie des jets, le second à la recherche de nouvelle physique.

\medskip

Les propriétés d'un jet (énergie, impulsion, charge) sont déterminées à l'aide de ses constituants. Ainsi, s'il en manque certains, ou si leurs énergies sont mal déterminées, les propriétés du jet seront impactées. Plusieurs effets sont à prendre en compte lorsque l'on veut corriger l'énergie des jets. Les différentes corrections utilisées par CMS ont été présentées dans le chapitre 4, l'accent étant principalement mis sur les corrections à l'aide d'événements \Pgamma + jets. En utilisant l'excellente reconstruction des photons pour contraindre l'énergie des jets, ces événements sont de puissants outils pour déterminer les corrections, utilisées globalement par toute la collaboration CMS, et plus particulièrement dans les analyses présentées dans la deuxième partie du manuscrit. Au final, l'incertitude systématique totale sur ces corrections est inférieure à \SI{2}{\%}, pour des jets de \SI{100}{\GeV} dans le tonneau.

\bigskip

La deuxième partie de ce manuscrit est consacrée à la recherche de nouvelle physique dans le secteur du quark top. En effet, de nombreux modèles prédisent de nouvelles particules se couplant préférentiellement au quark top, se manifestant comme des résonances dans le spectre de masse invariante \ttbar. Les outils de reconstruction de la masse invariante ont été présentés dans le chapitre 6.

\smallskip

La recherche de nouvelles particules de spin 1, se manifestant comme des résonances dans le spectre \mtt, a été présentée dans le chapitre 7. Dans cette analyse, le bruit de fond est estimé directement sur les données, permettant de réduire les incertitudes systématiques. L'absence de déviation par rapport aux prédictions du Modèle Standard est convertie en limite d'exclusion sur la section efficace de production de nouvelle physique. Pour un modèle topcolor prédisant un \zprime, boson neutre massif de spin 1, des masses inférieures à \SI{2.1}{\TeV} pour l'hypothèse de résonances étroites et \SI{2.7}{\TeV} pour l'hypothèse de résonances larges sont exclues. Des excitations de Kaluza-Klein du gluon, dans le modèle Randall-Sundrum, sont également exclues pour une masse inférieure à \SI{2.5}{\TeV}. Les résultats de cette étude ont été publiés dans \emph{Physical Review Letters}.

\smallskip

Le dernier chapitre est consacré à la recherche de particules de spin 0, assimilables à des bosons de Higgs massifs. Dans cette étude, les effets de l'interférence avec la production du Modèle Standard sont considérés, entraînant plusieurs difficultés, de la génération du signal jusqu'à la stratégie d'analyse. Les résultats sont encore préliminaires, et la sensibilité de l'analyse n'est pas encore suffisante pour pouvoir prétendre à une observation. La meilleure signification statistique attendue atteinte est d'environ \num{1.5}$\sigma$, pour une particule pseudo\-/scalaire de basse masse. De nombreuses pistes sont évoquées afin d'améliorer l'analyse.

\bigskip

Le LHC est actuellement arrêté pour environ 2 ans. Cet arrêt planifié permet de préparer le LHC au passage à \SI{14}{\TeV}. Il est aussi mis à profit afin de mettre à jour les détecteurs. Les collisions \Pproton{}\Pproton{} pour la physique devraient reprendre en avril 2015. Les deux analyses présentées dans ce manuscrit bénéficieront directement de la montée en puissance du LHC. En effet, les sections efficaces de production de nouvelle physique évoluent généralement plus rapidement que celles du Modèle Standard, et la luminosité collectée totale devrait avoisiner les \SI{100}{\invfb}. La sensibilité des analyses augmentera, permettant ainsi de sonder la présence de nouvelle physique à des échelles d'énergies inédites.